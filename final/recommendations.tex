\newpage

\section{Recommendations}
\label{sec:recommendations}

\subsection{Improved Phase Measurement}

Our current setup assumes that the real projection of the recieved signal tells us everything we need to know, and that adjusting the phase until the output signal is maximised gives us what we need.

However, the physics of the modulation transfer scheme are poorly understood.  It is entirely possible that the recieved signal contains some interesting elements in the imaginary axis.  We have not measured this, as we did not possess a reliable way of adjusting the phase of our signal.

\subsection{EOM driving frequencies}

The features in the Rb85 transitions are roughly 30MHz apart.  With a 20MHz EOM driver signal, we notice that the fringes of two strong signals overlap, crossing zero at a crossover.  If a 30MHz EOM driver were used instead, these two signals might constructively amplify, creating a very clean error signal.

An alternative approach uses the Rb87 features, which are generally more separated from one another.  Since our oscillator was fixed in frequency, we didn't get to experiment with the effect of the driving frequency on the quality of the error signal.

\subsection{Higher Power}

The best quality measurement we made was using 100\% of our laser's power.  Thus, it is possible that by increasing the power even further, an even-better signal-to-noise ratio could be achieved.

\subsection{Improved pre-amplifiers}

Near the end of the project, we decided that reducing noise was our priority.  The first and foremost concern is the noise from the amplifiers themselves.  The amplifiers we used, the ZFL-1000+, also exists in a low noise variant, the ZFL-1000LN+.  This amplifier has nearly identical properties, but the noise figure is lower, 3dB instead of 6dB \footnote{Noise figure is the amount that the SNR is increased by the amplifier.  Given a -20dBm carrier, with -30dBm of noise, and a +20dB amplifier with +6dB noise figure, the output signal will contain a 0dBm carrier, and -14dBm of noise, changing the SNR from 10dB to 4dB.}. The best locking features in \textbf{Appendices \ref{app:85pwr} - \ref{app:87temp}} have an SNR of approximately 5:1. The existing AOM signal (\textbf{Figure \ref{fig:aom_spectra}}), has an SNR of approximately 10:1. Switching out the existing amplifiers for a better model will increase the SNR by 6dB, to a value of 20:1 (two amplifiers in series). This is more robust than the existing AOM solution.

Additionally, the amplifiers used here do not have an optimal gain-bandwidth specification.  They are the most appropriate choice given the supplier (Mini-Circuits), but they are over-specified.  These amplifiers are designed to work up to 1000 MHz, whereas an ideal cutoff would be in the range of 25-30 MHz.  There are likely amplifiers better suited to this purpose.  If not, then one could probably be made which meets these specifications, and has very good performance.  See, for example \cite{ti_amps}.

\subsection{Additional Low-Pass Filtering}

A spectral analysis of our noise indicates that the energy distribution is flat in frequency, white noise, up to our low-pass cutoff frequency (10MHz).  A crude, but likely effective way to remove noise would be to bring in a lower low-pass filter, which would have the effect of "averaging" the error signal in time, reducing the amount of white noise, at the cost of response time.



