\newpage
\section*{Executive Summary}

This project, sponsored by Dr. Kirk Madison of the UBC Quantum Degenerate Gases Laboratory, invovled the construction and characterization of a robust laser frequency locking system based on the Pound-Drever-Hall locking method. \\

Existing systems at the sponsors' laboratory are already based on the PDH method, but use a slightly different approach.  Both the existing and proposed systems lock to a vapour cell.  The existing locking unit uses an acousto-optic modulator, which uses acoustics to produce a frequency-shifted diffraction pattern.  This pattern is filtered through a vapour cell, and then coupled into a photosensor.  From there, frequency-shifted parts are extracted, which produces an error signal.  This error signal locks the diode laser's control system to the desired transition resonance. The existing system is severaly limited in that the error signal has poor stability and is subject to various modulation effects which change the location of it's null locking points. \\

The new, EOM based, feedback system built here addresses these issues and hints at further improvements. A table-based optical setup based around modulation transfer spectroscopy was constructed and a Pound-Drever-Hall type error signal was generated using a photosensor and discrete RF components. The resulting signal has improved stability and non-drifting locking points. The resulting SNR is worse in the new system by a factor of two due to suboptimal component choices. However, simple change to more robust, readily available, electrical components will improve this by a factor of four, at minimum. \\

Closed loop testing to determine linewidth and carrier stability improvements was not possible due to time constraints, difficulty, and equipment availability. The sponsor was supplied with a full set of characterization data to specify the optimal configuration and is well poised to reconstruct this system on their master table at their convenience.

