\newpage
\section{Project Deliverables}

\subsection{EOM-based Optical Setup}

The sponsor's requirement of an optical pipeline including an EOM unit, Rubidium vapour cell and photosensor coupling was fully met. Due to the difficulty of creating a precise optical pipeline and this group's relative inexperience, it was decided to pursuse existing home-built or pre-built solutions whenever possible.

\subsection{Fast Analog Laser Control Feedback Circuit}

Constructed single channel demodulation.

\subsection{New vs. Old System Characterization}

Originally, the sponsor had suggested closed-loop testing of this locking scheme. With the use of a Fabry-Pi{\'e}rot interferometer, a direct comparison of laser linewidths using the old and new locking systems could be made. Unfortunately, due to timing, it was not possible to access, and make large changes to, the master laser table when this new system neared completion as the Madison Group researchers began experimental procedures which could not be easily interrupted. Additionally, it would have been either time-consuming or suboptimal to route either the optical signal or electrical signal across the room from our setup to the laser servo racks. The sponsor, Dr. Kirk Madison, directed us to focus on characterizing the error signal performance using the following metrics:
\begin{itemize}
    \item Error signals as a function of pump/probe power
    \item Error signals as a function of temperature
    \item Slope (mV/MHz) around locking points
    \item Noise (mVpp) around locking points
\end{itemize}
Given this data, it is intended that Madison Group researchers use this data to rebuild this setup on the master laser optical table at their earliest convenience.