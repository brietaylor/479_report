\newpage
\section*{Executive Summary}

This project, sponsored by the UBC Quantum Degenerate Gases
Laboratory---Madison Group, aims to build and characterize a high
fidelity, small linewidth, laser frequency locking system based on the
Pound-Drever-Hall locking method. \\

Existing systems at the sponsors' laboratory are already based
on the PDH method, but use a slightly different approach.  Both the
existing and proposed systems lock to a vapour cell.  The
existing locking unit uses an acousto-optic modulator, which uses acoustics
to produce a frequency-shifted diffraction pattern.  This pattern is filtered
through a vapour cell, and then coupled into a photosensor.  From there,
frequency-shifted parts are extracted, which gives us an error signal.  This
error signal locks the diode laser's control system to the resonant frequency
of the chosen element. \\

The existing acousto-optic modulator, however, at the desired power level, is limited to a modulation frequency of about 200 kHz. After filtering and processing, this results in a system response speed  on the order of 10kHz. This response speed has a direct impact on the resultant linewidth of the laser. \cite{madison14}  \\

Electro-optic modulators use the Pockels Effect, and do not do not experience the same limitations. A Pound-Drever-Hall locking system using an electro-optic modulator will
be built, and its performance will be compared to the existing acoustic-based
locking system. There will be three stages to this project:
\begin{enumerate}
 \item Measuring the operating parameters of the existing system.
 \item Building the electro-optic modulator based system.
 \item Measuring the operating parameters of the new system, using the
 old system as a benchmark.
\end{enumerate}
This project is technically demanding, but the technical expertise of this
group, as well as close collaboration with the sponsor, should result in
successful completion.

