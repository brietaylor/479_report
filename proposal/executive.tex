\newpage
\section*{Executive Summary}

This project, sponsored by the UBC Quantum Degenerate Gases
Laboratory - Madison Group, aims to build and characterize a high
fidelity, small linewidth, laser frequency locking system based on the
Pound-Drever-Hall (PDH) locking method. The QDG laboratory 
specializes in manipulation of atomic and molecular gases with
optical systems that implement some form of laser cooling. To 
efficiently cool or manipulate these ensembles, the lasers involved
must be tuned to specific frequencies, often corresponding to atomic
transitions, and have as small a linewidth as possible. Any increase 
in the fidelity of the probing or pumping lasers often has immediate
impact on the quantity of atoms that can be trapped/cooled, and the 
lowest attainable temperature in a trap. Specifically, in some cases, 
the linewidth must be as small as possible around the resonant frequency 
of a transition. \\

Existing systems at the sponsors' laboratory are already based 
on the PDH method, but use a slightly different approach. Both the 
existing and proposed systems will lock to a frequency-discriminating 
object, specifically a vapor cell of the relevant atomic species. The
existing locking unit makes use of an Acousto-optic Modulator (AOM), a 
device which takes a seeding beam from a diode laser as an input and, 
through acoustic vibration generated by a piezo-electric transducer, 
produces a Doppler-shifted diffraction pattern. This pattern is then 
sent through a vapor cell (though commonly, an external cavity is used), 
and the filtered beam is coupled into a photosensor and mixed/processed 
to produce an error signal. This signal is then used by the diode laser 
control system to lock to the resonance frequency of the selective 
element. Unfortunately, due to mechanical limitations of the AOM device, 
the dithering frequency of the modulated beam output, which drives the 
resolution of the error signal, is quite small, approximately 200 kHz. 
Furthermore, the processing required to use this signal decreases the 
settling time further, approximately 10-20 fold to 10 kHz [maybe cite 
something here, rather than relying on just Kirk's description]. This 
results in a large setting time for the control circuit, which 
necessarily increases the linewidth of the laser.  \\

To increase the bandwidth of the locking circuit and therefore 
reduce the linewidth of the master laser, a PDH unit based around an 
Electro-Optic Modulator (EOM) will be built, and its performance will be 
benchmarked to the existing AOM locking unit. An EOM, commonly known as 
a "Pockels Cell" as it exploits the Pockels Effect, is a physical medium 
which phase modulates a beam in its direction of propagation at a 
driving frequency. The output beam has the same fundamental as the input 
beam, as well as a well-defined spectrum of side-bands. This composite 
beam is then sent through the frequency selective unit and, again, mixed 
to produce an error signal. The primary difference, relative to the AOM 
unit, is that the mixing technique is slightly different and, more 
importantly, that the dithering frequency is significantly higher. Some 
EOM units can handle modulation at over 100 MHz, though, for this 
project, a frequency of 10-20 MHz will likely be sufficient. This 
drastic increase in the control loop bandwidth will result in a much 
smaller linewidth for the master laser, and, ultimately, higher 
experimental efficiency. \\

There will be three primary stages to this project. First, the 
existing system will be evaluated for key parameters such as laser 
linewidth, and the relationships between factors like the noise floor,
modulation frequency and the transfer function of the frequency selective
unit will be quantified and recorded as benchmarks. This information 
will be used to spec various optical and electronic components for the
locking unit. Second, the locking unit will be designed to fit on an 
optical breadboard of convenient size. It is possible that a custom CCA 
will have to be designed to control the EOM, generate the error signal, 
and interface with laboratory computers. Third, the built locking unit 
will be tested and benchmarked against the metrics obtained from the 
existing setup. The target linewidth of the master laser will be determined during the first design stage.
 
