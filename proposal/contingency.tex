\newpage
\section{Risks and Contingency}

Though the premise of this project, an increased-fidelity PDH error-signal
generator, may seem straightforward, there are several nuances which may
complicate development. The models presented in \textbf{Section
\ref{sec:theory}} are meant to give a qualitative overview of why this upgrade
is justified, but they ignore or simplify many parameters.

The signal processing here is mostly considered with the assumption that all
mixing or amplification is linear, and that signal attenuation, if present,
is minimal. In reality, the propagation of the beat-note signals through the
processing electronics may be more complex, with saturation effects and other
nonlinearities coming into effect. The incident side-band beat-note signals,
for example, may be subject to saturation effects in the phototransistor depending
on the incident laser power. This, and other effects, will have to be carefully
accounted for during electrical design. If something is overlooked, it may be that
tuning the locking unit is more difficult than anticipated.

As mentioned, another point of contention is the operation of the EOM unit itself.
Choosing between a readily-available but small-bandwidth low-power EOM which
is tuned around some pre-included oscillator, or developing a custom high-power,
wide-bandwidth driver will vastly change the amount of work that must be done to
arrive at a functional unit. Time spent designing an EOM controller, estimated
to take 1-2 weeks, would alternatively be used on assembly and benchmarking.

The primary detractor to how complete this project will be at delivery
is the amount of custom solutions (CCAs, custom software, etc) that must be
built in-house. Presumably, if the entire control loop can be built from
off-the-shelf components from, for example, Thorlabs and Mini-Circuits,
then a significant amount of time will be devoted to fine-tuning and
benchmarking a ready-to-use assembly. The sponsor has requested two functional, benchmarked PDH reference units at project completion. It is inconcievable that
these units will not be, at the very least, fully built during this term. Lack of
built PDH reference units, consisting of the opto-electronic assembly and electronic
mixing components, by 2014-12-20 is considered a project failure. The sponsor
has allowed for testing and benchmarking to be completed after delivery, if
absolutely necessary.