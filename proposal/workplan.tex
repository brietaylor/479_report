\newpage
\section{Work Plan}

It should be noted that this project is quite complex in nature, and that
any estimates here are likely to change during development. Of particular
concern are any steps which require significant debugging or integration with
existing experimental instruments.

\subsection{Milestones} %%%%%%%%%%%%%%%%% MILESTONES

The following milestones have been tenatively agreed to by both sponsor and
team members.

\begin{enumerate}
  \item{\textbf{Principal Schematics, BOMs} - 2014-10-12}
  - Subunit schematics
  and block diagrams must be completed. Optical assembly schematics may be
  graphical (not a specific CAD format) but must be properly labeled and
  included with the relevant physical models referenced. Electrical schematics
  for any CCAs should be completed, reviewed and approved. All
  relevant BOMs should be compiled and be ready to order. Lead times should be
  minimized as much as possible (1-3 weeks maximum), as there are no
  anticipated long-lead components. In this time, the existing AOM locking
  circuit must also be benchmarked as discussed.
  \item{\textbf{Mechanical Design, PCBs Finalized} - 2014-10-31}
  - Lead times for PCBs have historically been no longer than 1-2 weeks. Should
  CCAs have to be designed
  for EOM control and signal mixing, time will be spent here doing actual PCB
  layout and designing mechanical enclosures, with cooling options, if
  necessary. At this milestone, all designs should be approved and ready to
  order or be in the process of being ordered.
  \item{\textbf{Principal Assembly} - 2014-11-21}  This milestone is for a
  fully \emph{assembled} servomechanism, with all optical, electro-optical and
  electrical assemblies built and ready for testing. During this time,
  basic electrical testing - ensuring zero short-circuit conditions and correct
  inter-net connectivity - should be complete, with specific functional testing
  to be completed later. All optical elements should be assembled on an optical
  breadboard and be roughly calibrated such that beam propagation is close to
  ideal.
  \item{\textbf{Final Unit Delivery} - 2014-12-14} Upon delivery, the
  new PDH unit should ideally be fully built, functionally verified and
  benchmarked against the existing solution. At this stage the units would be
  ready for use in experiments.
\end{enumerate}

\subsection{Task Schedule}  %%%%%%%%%%%%% SCHEDULE

The proposed milestone schedule is broken up into tasks ([Milestone\#-Task\#]).
These are estimates of work that must be completed and may be subject to
alteration. The specifics behind each of these tasks are as follows (with
completion time estimates in brackets):

\begin{packed_itemize}
  \item{\textbf{[1a] Completion and Approval of Physical Models} (24 hrs)}
  - This constitutes a well-formatted block diagram of every unit in the servo
  loop as well as associated physical models. The format of the error signal
  should be explicit and well defined with respect to physical parameters (
  actual equations). This model is to be used as a reference for physical
  design and must be approved by the sponsor. If mandated, time will be spent on
  fully developing the quantum mechanical saturated absorption profile to
  determine the optimal phase modulation frequency.
  \item{\textbf{[1b] Electrical Schematic*} (24 hrs)}
  - A controller CCA may have to be designed and built. At this stage,
  the electrical schematic, with all of the components necessary for controlling
  the EOM and mixing the error signal. This will consist of, at the very least
  a VCO up to 100MHz, a VCO controller (if needed), the lab computer digital
  interface and analog mixing circuits (to be deterimined).
  \item{\textbf{[1c] BOM compilation and Component Ordering} (12hrs)}
  - This is potentially a significant step if any CCAs are to be designed.
  Components will have to be sourced and the requisite invoices will have to be
  produced so that the Madison Lab can put an order through. Depending on the
  number of suppliers involved, this step may represent a significant amount of
  paperwork/communication, and is not inconsequential.
  \item{\textbf{[1d] Evaluation of Existing AOM Servo} (36 hrs)}
  - The existing AOM based PDH unit is to be benchmarked for parameters such as
  its noise floor and slope around the locking point, as well as the resulting
  linewidth of the master laser. These features can be evaluated with equipment
  in the Madison Group laboratory. Fiber coupled laser sources are readily
  accessible within the laboratory. However, it is unclear whether the PDH
  feedback unit in existance will be in use for ongoing experiments. If so,
  a separate unit will have to be quickly constructed. All of the components
  for this should be readily available wtihin the Madison Group laboratory.
  \item{\textbf{[2a] Mounting Hardware}} (12 hrs)
  - It is likely that some mounting hardware will have to be constructed for
  various controllers and optical units to sit in either a standard equipment
  rack or near electro-optical elements on an optical table. Whether the EOM
  driver or FO receiver can be coupled to with long coaxial cables or fiber is
  dependent on how much attenuation is tolerable.
  \item{\textbf{[2b] Possible Signal Processing / EOM Controller CCA*} (24 hrs)}
  - If it is not possible to construct the EOM driving and signal mixing
  electronics from a collection of discrete RF components, a custom PCB will
  be designed and assembled in-house. This will likely not take long, discounting
  any difficulties in the electrical design stage, as group members have
  significant experience with PCB construction and layout. In this stage, actual
  board layout will likely take less than 12 hours.
  \item{\textbf{[3a] Assembly of Optical Breadboard} (6 hrs)}
  - Once acquired, all optical and electro-optical components must be assembled
  on an optical breadboard. This is simply a plate of moderate size with a
  grid of threaded holes to fasten standard optical elements to. The equipment
  required for this is avilable in the Madison lab. This is not expected to
  be difficult, but it may take some time to carefully lay out.
  \item{\textbf{[3b] Soldering and Assembly of custom CCAs*} (6 hrs)}
  - Should any custom CCAs be required, time will be spent soldering their
  components and doing basic electrical testing to ensure no short-circuit
  conditions and correct inter-net connectivity, etc.
  \item{\textbf{[3c] Manufacture and Assembly of Cabling/Interconnects} (3 hrs)}
  - Cabling must be created for connecting to laboratory power supplies
  and powering various components. Manually fabricated cabling will likely be
  standard braided cable with various MOLEX connectors. Any high-speed
  (100MHz +) cabling will likely be pre-purchased to maintain quality.
  \item{\textbf{[3d] Final Comprehensive Assembly} (12 hrs)}
  - This stage involves integrating all previously assembled subunits into their
  laboratory configuration. This will involve placing the optical assembly,
  placing any controlling electronics, connecting these components and then
  integrating them with any test equipment or lab systems.
  \item{\textbf{[4a] Unit Block Validation} (12hrs)}
  - All subunits will be quickly tested to ensure that they produce signals that
  are at least qualitatively correct. This is considered the "trial run" as it
  will hopefully identify any major problems, such as overattenuation,
  excessive power draw, excessive noise or instability, etc. The servomechanism
  will be evaluated piecewise to ensure that component feedthrough is correct.
  \item{\textbf{[4b] Comprehensive Benchmarking} (24hrs)}
  - This step mirrors step [1d], but for the presumably-finished EOM-based
  locking unit. Identical parameters will be extracted, and compared with
  those of the AOM solution, to formally establish improvement metrics.
  \item{\textbf{[4c] Fine Tuning / Final Configuration} (???, 24hrs ++)}
  - With current background knowledge and understanding, it is not possible
  to predict how long tuning the system will take. While the scope and
  technical background behind the project is understoood, anticipating
  complications that may arise from problems such as component nonlinearities or
  induced noise is not currently possible. This is expected to be the
  longest single task of the project.
\end{packed_itemize}

Tasks marked with (*) are potentially unnecessary, should ready-made solutions
be found. The total estimated task time, barring final debugging, sums to 180
hours over the term. The testing and assembly estimates are likely accurate, as
they are based on previous experience. However it is likely that the model
development and testing will exceed the estimated time significantly as this
project is based on complex physical principles and has high technical demands.
The final debugging step is estimated to take the most amount of time and effort
as there is potential for significant problems to arise. Fixing these problems
will require some re-design and re-assembly, followed by further testing. This
testing, in particular, has the potential to take a long time, especially if
significant lab resources are required.

\subsection{Team Responsibilities} %%%%%%% RESPONSIBILITY

As per the proposal guidelines, the following team members are assigned the
suggested team roles:
\begin{packed_itemize}
  \item{\textbf{Steve Novakov} - Project Manager}
    \begin{packed_itemize}
      \item shall maintain and ensure project schedule
      \item shall submit proposal and recommendation report and delegate their
      workload
      \item shall maintain communications with sponsor and schedule meetings,
      when necessary
    \end{packed_itemize}
  \item{\textbf{Jeff Taylor} - Technical Manager}
    \begin{packed_itemize}
      \item shall submit weekly reports and maintain project technical data
      \item shall drive testing and validation efforts
    \end{packed_itemize}
\end{packed_itemize}

The nature of the project is such that both members shall be conducting
investigations and driving design schedules, but the roles are explicitly
delegated for formality, with the designated members having final oversight
and providing approval. The sponsor shall interact with both team members on a
regular basis making the single liason role effectively meaningless.

\subsection{Sponsor Interactions}  %%%%%%% SPONSOR INTERACTIONS

With respect to communicating with the sponsor, Dr. Kirk Madison, there shall
at least be one weekly email sent every Sunday night summarizing work that has
been completed and upcoming tasks in a relevant timeframe. These shall mirror
the content of the weekly 479 reports, but will be composed with the sponsor in
mind (more specifics, more technical jargon). During any stages where there
is a need for interaction with Madison Group laboratory equipment, (e.g. during
testing and benchmarking, the appropriate lab staff, likely a graduate
student or Dr. Madison himself, shall be consulted with to approve and possibly
assist with the procedure.

\subsection{Resources and Budget}

At this time, it is possible to identify several high-level items (with lack of
specificity about sub-assemblies) which will be necessary to construct the PDH
unit. These items, with some commentary on status and price,
can be reviewed in \textbf{Table \ref{budget_table}}

\begin{table}[!hrt]
  \begin{tabularx}{\linewidth}{|L|L|L|}
  \hline
  \textbf{Item} & \textbf{Status} & \textbf{Approx. Purchase Cost (source)} \\
  \hline
  Rubidium Vapour Cell & exists in inventory & \$500 (Thorlabs) \\
  EO Phase Modulator & exists in inventory & \$2,500 (Thorlabs) \\
  High-speed FO Receiver & exists in inventory & \$200 (custom) \\
  PDH Controller CCA & as necessary & $>$\$200 (custom) \\
  Discrete RF Compoonents & as necessary &$>$\$200 (Mini-Circuits) \\
  Cabling and Accessories & as necessary &$>$\$200 (Mini-Circuits) \\
  \hline
  \end{tabularx}
  \caption{Brief overview of major subcomponents and their estimated status,
  with respect to acquisition. Items stated to "exist in inventory" are likely
  available for use from the Madison Lab, but are allowed to be purchased, if
  necessary. Price estimates may represent an amalgamation of
  components from various vendors.}
  \label{budget_table}
\end{table}

There is no established hard limit on the budget for this project, but, as all
purchases must be approved by Dr. Madison, they are expected to be provably
reasonable and necessary. For example, should a new EOM unit need to be
purchased, it may cost anywhere in the neighbourhood of \$2-3000. With respect
to the PDH Controller CCA, which must be designed and built sometime this
term, the board and components will likely price in the \$2-300 range.
The Madison Group does have many optical components and optical breadboards
in storage and available for use. It is unlikely that any optics will have to
be purchased, again, unless provably necessary. It is possible that some
mounting hardware will have to be purchased or manufactured, but there is
allowance for small work jobs to be done by the UBC Hennings machine shop,
if necessary. Fiber optic receivers, with bandwidth up to 12 GHz are readily
available throughout the laboratory.
