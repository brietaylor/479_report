\newpage
\section{Work Plan}

\subsection{Milestones} %%%%%%%%%%%%%%%%% MILESTONES

The following milestones have been tenatively agreed to by both sponsor and
team members.

\begin{enumerate}
  \item{\textbf{Principal Schematics, BOMs} - 2014-10-05} - Subunit schematics
  and block diagrams must be completed. Optical assembly schematics may be
  graphical (not a specific CAD format) but must be properly labeled and
  included with the relevant physical models referenced. Electrical schematics
  for the PDH Controller CCA should be completed, reviewed and approved. All
  relevant BOMs should be compiled and be ready to order. Lead times should be
  minimized as much as possible (1-3 weeks maximum), as there are no
  anticipated long-lead components. In this time, the existing AOM locking
  circuit must also be benchmarked
  \item{\textbf{Mechanical Design, PCBs} - 2014-10-31} Lead times for PCBs have
  historically been no longer than 1-2 weeks.
  \item{\textbf{Principal Assembly} - 2014-11-21}
  \item{\textbf{Final Unit Test} - 2014-12-14}
\end{enumerate}

\subsection{Task Schedule}  %%%%%%%%%%%%% SCHEDULE

The proposed milestone schedule is broken up into tasks ([Milestone\#-Task\#]).
These are estimates of work that must be completed and may be subject to
alteration. The specifics behind each of these tasks are as follows (with
completion time estimates in brackets):

\begin{packed_itemize}
  \item{\textbf{[1a] Completion and Approval of Physical Models} (24 hrs)}
  - This constitutes a well-formatted block diagram of every unit in the servo
  loop as well as associated physical models. The format of the error signal
  should be explicit and well defined with respect to physical parameters (
  actual equations). This model is to be used as a reference for physical
  design and must be approved by the sponsor. If mandated, time will be spent on
  fully developing the quantum mechanical saturated absorption profile to
  determine the optimal phase modulation frequency.
  \item{\textbf{[1b] Electrical Schematic*} (24 hrs)}
  - A controller CCA may have to be designed and built. At this stage,
  the electrical schematic, with all of the components necessary for controlling
  the EOM and mixing the error signal. This will consist of, at the very least
  a VCO up to 100MHz, a VCO controller (if needed), the lab computer digital
  interface and analog mixing circuits (to be deterimined).
  \item{\textbf{[1c] BOM compilation and Component Ordering} (12hrs)}
  \item{\textbf{[1d] Evaluation of Existing AOM Servo} (12 hrs)}
  \item{\textbf{[2a] Mounting Hardware}} (12 hrs)
  \item{\textbf{[2b] PDH Controller CCA} (12 hrs)}
  \item{\textbf{[3a] Assembly of Optical Breadboard} (6 hrs)}
  \item{\textbf{[3b] Soldering and Assembly of PDH Controller CCA} (3 hrs)}
  \item{\textbf{[3c] Manufacture and Assembly of Cabling/Interconnects} (3 hrs)}
  \item{\textbf{[3d] Final Comprehensive Assembly} (12 hrs)}
  \item{\textbf{[4a] Unit Block Validation} (12hrs)}
  \item{\textbf{[4b] Comprehensive Benchmarking} (24hrs)}
  \item{\textbf{[4c] Fine Tuning / Final Configuration} (???)}
\end{packed_itemize}

Tasks marked with (*) are potentially unnecessary, should ready-made solutions
be found. Total estimated task time sums to 144 hours over the term. The testing
and assembly estimates are likely accurate, as they are based on previous
experience. However it is likely that the model development and testing will
exceed the estimated time significantly as this project is based on complex
physical principles and has high technical demands.

Mention here that there may have to be some final adjustments/fixes, etc, and
discuss what those might be and the timeline, but don't assign formal tasks to
them.

\subsection{Team Responsibilities} %%%%%%% RESPONSIBILITY

As per the proposal guidelines, the following team members are assigned the
suggested team roles:
\begin{packed_itemize}
  \item{\textbf{Steve Novakov} - Project Manager}
    \begin{packed_itemize}
      \item shall maintain and ensure project schedule
      \item shall submit proposal and recommendation report and delegate their
      workload
      \item shall maintain communications with sponsor and schedule meetings,
      when necessary
    \end{packed_itemize}
  \item{\textbf{Jeff Taylor} - Technical Manager}
    \begin{packed_itemize}
      \item shall submit weekly reports and maintain project technical data
      \item shall drive testing and validation efforts
    \end{packed_itemize}
\end{packed_itemize}

The nature of the project is such that both members shall be conducting
investigations and driving design schedules, but the roles are explicitly
delegated for formality, with the designated members having final oversight
and providing approval. The sponsor shall interact with both team members on a
regular basis making a single liason role untenable.

\subsection{Sponsor Interactions}  %%%%%%% SPONSOR INTERACTIONS

With respect to communicating with the sponsor, Dr. Kirk Madison, there shall
at least be one weekly email sent every Sunday night summarizing work that has
been completed and upcoming tasks in a relevant timeframe. These shall mirror
the content of the weekly 479 reports, but will be composed with the sponsor in
mind (more specifics, more technical jargon). During any stages where there
is a need for interaction with Madison Group laboratory equipment, (e.g. during
testing and benchmarking, the appropriate lab staff, likely a graduate
student or Dr. Madison himself, shall be consulted with to approve and possibly
assist with the procedure.

\subsection{Resources and Budget}

At this time, it is possible to identify several high-level items (with lack of
specificity about sub-assemblies) which will be necessary to construct the PDH
feedback loop. These items, with some commentary on status and price,
can be reviewed in \textbf{Table \ref{budget_table}}

\begin{table}[!hrt]
  \begin{tabularx}{\linewidth}{|L|L|L|}
  \hline
  \textbf{Item} & \textbf{Status} & \textbf{Approx. Purchase Cost (source)} \\
  \hline
  Rubidium Vapour Cell & exists in inventory & \$500 (Thorlabs) \\
  EO Phase Modulator & exists in inventory & \$2,500 (Thorlabs) \\
  High-speed FO Receiver & exists in inventory & \$200 (custom) \\
  PDH Controller CCA & to be designed/built & $>$\$200 (custom) \\
  \hline
  \end{tabularx}
  \caption{Brief overview of major subcomponents and their estimated status,
  with respect to acquisition. Items stated to "exist in inventory" are likely
  available for use from the Madison Lab, but are allowed to be purchased, if
  necessary. Price estimates may represent an amalgamation of
  components from various vendors.}
  \label{budget_table}
\end{table}

There is no established hard limit on the budget for this project, but, as all
purchases must be approved by Dr. Madison, they are expected to be provably
reasonable and necessary. For example, should a new EOM unit need to be
purchased, it may cost anywhere in the neighbourhood of \$2-3000. With respect
to the PDH Controller CCA, which must be designed and built sometime this
term, the board and components will likely price in the \$2-300 range.
The Madison Group does have many optical components and optical breadboards
in storage and available for use. It is unlikely that any optics will have to
be purchased, again, unless provably necessary. It is possible that some
mounting hardware will have to be purchased or manufactured, but there is
allowance for small work jobs to be done by the UBC Hennings machine shop,
if necessary.
